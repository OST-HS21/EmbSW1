\section{Language Linkage}
Der Linker benötigt einen eindeutigen Namen für jede Funktion.
In C wird die Funktion \lstinline[style=cppstyle]{foo()} zu \lstinline[style=cppstyle]{_foo}.

Das Zuweisen eines eindeutigen Namens in C++ wird als Name Mangling bezeichnet.
Die Funktion \lstinline[style=cppstyle]{foo(int)} wird dann beispielsweise zu \lstinline[style=cppstyle]{_foo_i}. Die Funktion \lstinline[style=cppstyle]{foo(double, int)} wird dann beispielsweise zu \lstinline[style=cppstyle]{_foo_d_i}. \lstinline[style=cppstyle]{MyClass::foo(int)} zu \lstinline[style=cppstyle]{_MyClass_foo_i}.

Im Funktionsprototypen (C++‐seitig) kann die Language Linkage festgelegt werden:
\begin{lstlisting}[language=c, style=cppstyle]
extern "C" void foo(int);	// use C ll
extern void goo(int);		// use C++ ll
void hoo(int);				// use C++ ll
extern "C++" void koo(int);	// use C++ ll

extern "C"
{
	// list multiple prototypes with C linkage, or
	// #include C header file(s)
}

// Alternativ:
#ifdef __cplusplus
extern "C"
{
	#endif
	// list multiple prototypes with C linkage, or
	// #include C header file(s)
	#ifdef __cplusplus
}
#endif
\end{lstlisting}